% Generated by Sphinx.
\def\sphinxdocclass{report}
\documentclass[letterpaper,10pt,english]{sphinxmanual}
\usepackage[utf8]{inputenc}
\DeclareUnicodeCharacter{00A0}{\nobreakspace}
\usepackage{cmap}
\usepackage[T1]{fontenc}
\usepackage{babel}
\usepackage{times}
\usepackage[Bjarne]{fncychap}
\usepackage{longtable}
\usepackage{sphinx}
\usepackage{multirow}


\title{ps.py Documentation}
\date{March 31, 2013}
\release{0.0.1}
\author{alpacas}
\newcommand{\sphinxlogo}{}
\renewcommand{\releasename}{Release}
\makeindex

\makeatletter
\def\PYG@reset{\let\PYG@it=\relax \let\PYG@bf=\relax%
    \let\PYG@ul=\relax \let\PYG@tc=\relax%
    \let\PYG@bc=\relax \let\PYG@ff=\relax}
\def\PYG@tok#1{\csname PYG@tok@#1\endcsname}
\def\PYG@toks#1+{\ifx\relax#1\empty\else%
    \PYG@tok{#1}\expandafter\PYG@toks\fi}
\def\PYG@do#1{\PYG@bc{\PYG@tc{\PYG@ul{%
    \PYG@it{\PYG@bf{\PYG@ff{#1}}}}}}}
\def\PYG#1#2{\PYG@reset\PYG@toks#1+\relax+\PYG@do{#2}}

\expandafter\def\csname PYG@tok@gd\endcsname{\def\PYG@tc##1{\textcolor[rgb]{0.63,0.00,0.00}{##1}}}
\expandafter\def\csname PYG@tok@gu\endcsname{\let\PYG@bf=\textbf\def\PYG@tc##1{\textcolor[rgb]{0.50,0.00,0.50}{##1}}}
\expandafter\def\csname PYG@tok@gt\endcsname{\def\PYG@tc##1{\textcolor[rgb]{0.00,0.27,0.87}{##1}}}
\expandafter\def\csname PYG@tok@gs\endcsname{\let\PYG@bf=\textbf}
\expandafter\def\csname PYG@tok@gr\endcsname{\def\PYG@tc##1{\textcolor[rgb]{1.00,0.00,0.00}{##1}}}
\expandafter\def\csname PYG@tok@cm\endcsname{\let\PYG@it=\textit\def\PYG@tc##1{\textcolor[rgb]{0.25,0.50,0.56}{##1}}}
\expandafter\def\csname PYG@tok@vg\endcsname{\def\PYG@tc##1{\textcolor[rgb]{0.73,0.38,0.84}{##1}}}
\expandafter\def\csname PYG@tok@m\endcsname{\def\PYG@tc##1{\textcolor[rgb]{0.13,0.50,0.31}{##1}}}
\expandafter\def\csname PYG@tok@mh\endcsname{\def\PYG@tc##1{\textcolor[rgb]{0.13,0.50,0.31}{##1}}}
\expandafter\def\csname PYG@tok@cs\endcsname{\def\PYG@tc##1{\textcolor[rgb]{0.25,0.50,0.56}{##1}}\def\PYG@bc##1{\setlength{\fboxsep}{0pt}\colorbox[rgb]{1.00,0.94,0.94}{\strut ##1}}}
\expandafter\def\csname PYG@tok@ge\endcsname{\let\PYG@it=\textit}
\expandafter\def\csname PYG@tok@vc\endcsname{\def\PYG@tc##1{\textcolor[rgb]{0.73,0.38,0.84}{##1}}}
\expandafter\def\csname PYG@tok@il\endcsname{\def\PYG@tc##1{\textcolor[rgb]{0.13,0.50,0.31}{##1}}}
\expandafter\def\csname PYG@tok@go\endcsname{\def\PYG@tc##1{\textcolor[rgb]{0.20,0.20,0.20}{##1}}}
\expandafter\def\csname PYG@tok@cp\endcsname{\def\PYG@tc##1{\textcolor[rgb]{0.00,0.44,0.13}{##1}}}
\expandafter\def\csname PYG@tok@gi\endcsname{\def\PYG@tc##1{\textcolor[rgb]{0.00,0.63,0.00}{##1}}}
\expandafter\def\csname PYG@tok@gh\endcsname{\let\PYG@bf=\textbf\def\PYG@tc##1{\textcolor[rgb]{0.00,0.00,0.50}{##1}}}
\expandafter\def\csname PYG@tok@ni\endcsname{\let\PYG@bf=\textbf\def\PYG@tc##1{\textcolor[rgb]{0.84,0.33,0.22}{##1}}}
\expandafter\def\csname PYG@tok@nl\endcsname{\let\PYG@bf=\textbf\def\PYG@tc##1{\textcolor[rgb]{0.00,0.13,0.44}{##1}}}
\expandafter\def\csname PYG@tok@nn\endcsname{\let\PYG@bf=\textbf\def\PYG@tc##1{\textcolor[rgb]{0.05,0.52,0.71}{##1}}}
\expandafter\def\csname PYG@tok@no\endcsname{\def\PYG@tc##1{\textcolor[rgb]{0.38,0.68,0.84}{##1}}}
\expandafter\def\csname PYG@tok@na\endcsname{\def\PYG@tc##1{\textcolor[rgb]{0.25,0.44,0.63}{##1}}}
\expandafter\def\csname PYG@tok@nb\endcsname{\def\PYG@tc##1{\textcolor[rgb]{0.00,0.44,0.13}{##1}}}
\expandafter\def\csname PYG@tok@nc\endcsname{\let\PYG@bf=\textbf\def\PYG@tc##1{\textcolor[rgb]{0.05,0.52,0.71}{##1}}}
\expandafter\def\csname PYG@tok@nd\endcsname{\let\PYG@bf=\textbf\def\PYG@tc##1{\textcolor[rgb]{0.33,0.33,0.33}{##1}}}
\expandafter\def\csname PYG@tok@ne\endcsname{\def\PYG@tc##1{\textcolor[rgb]{0.00,0.44,0.13}{##1}}}
\expandafter\def\csname PYG@tok@nf\endcsname{\def\PYG@tc##1{\textcolor[rgb]{0.02,0.16,0.49}{##1}}}
\expandafter\def\csname PYG@tok@si\endcsname{\let\PYG@it=\textit\def\PYG@tc##1{\textcolor[rgb]{0.44,0.63,0.82}{##1}}}
\expandafter\def\csname PYG@tok@s2\endcsname{\def\PYG@tc##1{\textcolor[rgb]{0.25,0.44,0.63}{##1}}}
\expandafter\def\csname PYG@tok@vi\endcsname{\def\PYG@tc##1{\textcolor[rgb]{0.73,0.38,0.84}{##1}}}
\expandafter\def\csname PYG@tok@nt\endcsname{\let\PYG@bf=\textbf\def\PYG@tc##1{\textcolor[rgb]{0.02,0.16,0.45}{##1}}}
\expandafter\def\csname PYG@tok@nv\endcsname{\def\PYG@tc##1{\textcolor[rgb]{0.73,0.38,0.84}{##1}}}
\expandafter\def\csname PYG@tok@s1\endcsname{\def\PYG@tc##1{\textcolor[rgb]{0.25,0.44,0.63}{##1}}}
\expandafter\def\csname PYG@tok@gp\endcsname{\let\PYG@bf=\textbf\def\PYG@tc##1{\textcolor[rgb]{0.78,0.36,0.04}{##1}}}
\expandafter\def\csname PYG@tok@sh\endcsname{\def\PYG@tc##1{\textcolor[rgb]{0.25,0.44,0.63}{##1}}}
\expandafter\def\csname PYG@tok@ow\endcsname{\let\PYG@bf=\textbf\def\PYG@tc##1{\textcolor[rgb]{0.00,0.44,0.13}{##1}}}
\expandafter\def\csname PYG@tok@sx\endcsname{\def\PYG@tc##1{\textcolor[rgb]{0.78,0.36,0.04}{##1}}}
\expandafter\def\csname PYG@tok@bp\endcsname{\def\PYG@tc##1{\textcolor[rgb]{0.00,0.44,0.13}{##1}}}
\expandafter\def\csname PYG@tok@c1\endcsname{\let\PYG@it=\textit\def\PYG@tc##1{\textcolor[rgb]{0.25,0.50,0.56}{##1}}}
\expandafter\def\csname PYG@tok@kc\endcsname{\let\PYG@bf=\textbf\def\PYG@tc##1{\textcolor[rgb]{0.00,0.44,0.13}{##1}}}
\expandafter\def\csname PYG@tok@c\endcsname{\let\PYG@it=\textit\def\PYG@tc##1{\textcolor[rgb]{0.25,0.50,0.56}{##1}}}
\expandafter\def\csname PYG@tok@mf\endcsname{\def\PYG@tc##1{\textcolor[rgb]{0.13,0.50,0.31}{##1}}}
\expandafter\def\csname PYG@tok@err\endcsname{\def\PYG@bc##1{\setlength{\fboxsep}{0pt}\fcolorbox[rgb]{1.00,0.00,0.00}{1,1,1}{\strut ##1}}}
\expandafter\def\csname PYG@tok@kd\endcsname{\let\PYG@bf=\textbf\def\PYG@tc##1{\textcolor[rgb]{0.00,0.44,0.13}{##1}}}
\expandafter\def\csname PYG@tok@ss\endcsname{\def\PYG@tc##1{\textcolor[rgb]{0.32,0.47,0.09}{##1}}}
\expandafter\def\csname PYG@tok@sr\endcsname{\def\PYG@tc##1{\textcolor[rgb]{0.14,0.33,0.53}{##1}}}
\expandafter\def\csname PYG@tok@mo\endcsname{\def\PYG@tc##1{\textcolor[rgb]{0.13,0.50,0.31}{##1}}}
\expandafter\def\csname PYG@tok@mi\endcsname{\def\PYG@tc##1{\textcolor[rgb]{0.13,0.50,0.31}{##1}}}
\expandafter\def\csname PYG@tok@kn\endcsname{\let\PYG@bf=\textbf\def\PYG@tc##1{\textcolor[rgb]{0.00,0.44,0.13}{##1}}}
\expandafter\def\csname PYG@tok@o\endcsname{\def\PYG@tc##1{\textcolor[rgb]{0.40,0.40,0.40}{##1}}}
\expandafter\def\csname PYG@tok@kr\endcsname{\let\PYG@bf=\textbf\def\PYG@tc##1{\textcolor[rgb]{0.00,0.44,0.13}{##1}}}
\expandafter\def\csname PYG@tok@s\endcsname{\def\PYG@tc##1{\textcolor[rgb]{0.25,0.44,0.63}{##1}}}
\expandafter\def\csname PYG@tok@kp\endcsname{\def\PYG@tc##1{\textcolor[rgb]{0.00,0.44,0.13}{##1}}}
\expandafter\def\csname PYG@tok@w\endcsname{\def\PYG@tc##1{\textcolor[rgb]{0.73,0.73,0.73}{##1}}}
\expandafter\def\csname PYG@tok@kt\endcsname{\def\PYG@tc##1{\textcolor[rgb]{0.56,0.13,0.00}{##1}}}
\expandafter\def\csname PYG@tok@sc\endcsname{\def\PYG@tc##1{\textcolor[rgb]{0.25,0.44,0.63}{##1}}}
\expandafter\def\csname PYG@tok@sb\endcsname{\def\PYG@tc##1{\textcolor[rgb]{0.25,0.44,0.63}{##1}}}
\expandafter\def\csname PYG@tok@k\endcsname{\let\PYG@bf=\textbf\def\PYG@tc##1{\textcolor[rgb]{0.00,0.44,0.13}{##1}}}
\expandafter\def\csname PYG@tok@se\endcsname{\let\PYG@bf=\textbf\def\PYG@tc##1{\textcolor[rgb]{0.25,0.44,0.63}{##1}}}
\expandafter\def\csname PYG@tok@sd\endcsname{\let\PYG@it=\textit\def\PYG@tc##1{\textcolor[rgb]{0.25,0.44,0.63}{##1}}}

\def\PYGZbs{\char`\\}
\def\PYGZus{\char`\_}
\def\PYGZob{\char`\{}
\def\PYGZcb{\char`\}}
\def\PYGZca{\char`\^}
\def\PYGZam{\char`\&}
\def\PYGZlt{\char`\<}
\def\PYGZgt{\char`\>}
\def\PYGZsh{\char`\#}
\def\PYGZpc{\char`\%}
\def\PYGZdl{\char`\$}
\def\PYGZhy{\char`\-}
\def\PYGZsq{\char`\'}
\def\PYGZdq{\char`\"}
\def\PYGZti{\char`\~}
% for compatibility with earlier versions
\def\PYGZat{@}
\def\PYGZlb{[}
\def\PYGZrb{]}
\makeatother

\begin{document}

\maketitle
\tableofcontents
\phantomsection\label{index::doc}



\chapter{ps module}
\label{index:ps-py-documentation}\label{index:ps-module}\label{index:module-ps}\index{ps (module)}
This module contains facility classes and functions to
access PowerSchool and change the XML data into a more
workable format.
\index{Connection (class in ps)}

\begin{fulllineitems}
\phantomsection\label{index:ps.Connection}\pysigline{\strong{class }\code{ps.}\bfcode{Connection}}
Creates a temporary browser that connects to PowerSchool using HTTP.

\begin{Verbatim}[commandchars=\\\{\}]
\PYG{g+gp}{\PYGZgt{}\PYGZgt{}\PYGZgt{} }\PYG{n}{conn} \PYG{o}{=} \PYG{n}{Connection}\PYG{p}{(}\PYG{p}{)}
\PYG{g+gp}{\PYGZgt{}\PYGZgt{}\PYGZgt{} }\PYG{n}{conn}\PYG{o}{.}\PYG{n}{login}\PYG{p}{(}\PYG{l+s}{\PYGZsq{}}\PYG{l+s}{http://powerschool.ausd.net}\PYG{l+s}{\PYGZsq{}}\PYG{p}{,} \PYG{l+s}{\PYGZsq{}}\PYG{l+s}{12345}\PYG{l+s}{\PYGZsq{}}\PYG{p}{,} \PYG{l+s}{\PYGZsq{}}\PYG{l+s}{password}\PYG{l+s}{\PYGZsq{}}\PYG{p}{)}
\PYG{g+go}{True}
\PYG{g+gp}{\PYGZgt{}\PYGZgt{}\PYGZgt{} }\PYG{n}{student} \PYG{o}{=} \PYG{n}{conn}\PYG{o}{.}\PYG{n}{get\PYGZus{}student}\PYG{p}{(}\PYG{p}{)}
\PYG{g+gp}{\PYGZgt{}\PYGZgt{}\PYGZgt{} }\PYG{n}{student}\PYG{o}{.}\PYG{n}{first\PYGZus{}name}
\PYG{g+go}{\PYGZdq{}Johnny\PYGZdq{}}
\PYG{g+gp}{\PYGZgt{}\PYGZgt{}\PYGZgt{} }\PYG{n}{student}\PYG{o}{.}\PYG{n}{gpa}
\PYG{g+go}{\PYGZdq{}2.5\PYGZdq{}}
\end{Verbatim}
\index{close() (ps.Connection method)}

\begin{fulllineitems}
\phantomsection\label{index:ps.Connection.close}\pysiglinewithargsret{\bfcode{close}}{}{}
Closes the browser instance.

\end{fulllineitems}

\index{error (ps.Connection attribute)}

\begin{fulllineitems}
\phantomsection\label{index:ps.Connection.error}\pysigline{\bfcode{error}\strong{ = None}}
(string) error message from PowerSchool

\end{fulllineitems}

\index{get\_student() (ps.Connection method)}

\begin{fulllineitems}
\phantomsection\label{index:ps.Connection.get_student}\pysiglinewithargsret{\bfcode{get\_student}}{}{}~
\begin{notice}{note}{Note:}
Must be logged in to work.
\end{notice}

Create a student from the login information.

\end{fulllineitems}

\index{get\_xml() (ps.Connection method)}

\begin{fulllineitems}
\phantomsection\label{index:ps.Connection.get_xml}\pysiglinewithargsret{\bfcode{get\_xml}}{}{}~
\begin{notice}{note}{Note:}
Must be logged in to work.
\end{notice}

Downloads the XML data from PowerSchool and returns its contents.

\end{fulllineitems}

\index{logged\_in (ps.Connection attribute)}

\begin{fulllineitems}
\phantomsection\label{index:ps.Connection.logged_in}\pysigline{\bfcode{logged\_in}\strong{ = None}}
(bool) returns whether Connection was able to log in

\end{fulllineitems}

\index{login() (ps.Connection method)}

\begin{fulllineitems}
\phantomsection\label{index:ps.Connection.login}\pysiglinewithargsret{\bfcode{login}}{\emph{url}, \emph{username}, \emph{password}}{}
Logs into \emph{url} with credentials \emph{username} and \emph{password}.
\begin{quote}\begin{description}
\item[{Parameters}] \leavevmode\begin{itemize}
\item {} 
\textbf{url} (\emph{string}) -- PowerSchool URL to login into. ex. \href{http://powerschool.ausd.net/}{http://powerschool.ausd.net/}

\item {} 
\textbf{username} (\emph{string}) -- PowerSchool username (usually a Student ID\#)

\item {} 
\textbf{password} (\emph{string}) -- PowerSchool password

\end{itemize}

\item[{Returns}] \leavevmode
\textbf{True} if login was successful. Otherwise, \textbf{False}.

\end{description}\end{quote}

\end{fulllineitems}

\index{student (ps.Connection attribute)}

\begin{fulllineitems}
\phantomsection\label{index:ps.Connection.student}\pysigline{\bfcode{student}\strong{ = None}}
(Student) a Student from the Connection

\end{fulllineitems}

\index{xml\_data (ps.Connection attribute)}

\begin{fulllineitems}
\phantomsection\label{index:ps.Connection.xml_data}\pysigline{\bfcode{xml\_data}\strong{ = None}}
(string) downloaded XML data

\end{fulllineitems}


\end{fulllineitems}

\index{Student (class in ps)}

\begin{fulllineitems}
\phantomsection\label{index:ps.Student}\pysiglinewithargsret{\strong{class }\code{ps.}\bfcode{Student}}{\emph{loadstring=None}}{}
A Student is a data structure that holds:

\begin{tabulary}{\linewidth}{|L|L|L|}
\hline
\textbf{\relax 
Field
} & \textbf{\relax 
Type
} & \textbf{\relax 
Description
}\\\hline

first\_name
 & 
(string)
 & 
First name
\\\hline

last\_name
 & 
(string)
 & 
Last name
\\\hline

gender
 & 
(string)
 & 
Gender (Male/Female)
\\\hline

gpa
 & 
(string)
 & 
Grade point average
\\\hline

courses
 & 
(course{[}{]})
 & 
Courses
\\\hline
\end{tabulary}


A course is a dictionary containing:

\begin{tabulary}{\linewidth}{|L|L|L|}
\hline
\textbf{\relax 
Field
} & \textbf{\relax 
Type
} & \textbf{\relax 
Description
}\\\hline

name
 & 
(string)
 & 
Course name
\\\hline

teacher
 & 
(string)
 & 
Course teacher (LastName, FirstName)
\\\hline

letter\_grade
 & 
(string)
 & 
Letter grade received (A, B, C, etc.)
\\\hline

number\_grade
 & 
(string)
 & 
Number grade received (percentage)
\\\hline

in\_progress
 & 
(bool)
 & 
Sees if course is to be included
\\\hline

assignments
 & 
(assn{[}{]})
 & 
Assignments
\\\hline

categories
 & 
(string{[}{]})
 & 
Categories of assignments for course
\\\hline
\end{tabulary}


An assn (assignment) is a dictionary containing:

\begin{tabulary}{\linewidth}{|L|L|L|}
\hline
\textbf{\relax 
Field
} & \textbf{\relax 
Type
} & \textbf{\relax 
Description
}\\\hline

name
 & 
(string)
 & 
Assignment name
\\\hline

category
 & 
(string)
 & 
Category of assignment (test, project, etc.)
\\\hline

due\_date
 & 
(string)
 & 
Due date (Month/Day/Year format)
\\\hline

score
 & 
(float)
 & 
Score gotten on assignment
\\\hline

out\_of
 & 
(float)
 & 
Total amount of points possible
\\\hline
\end{tabulary}

\begin{quote}\begin{description}
\item[{Parameters}] \leavevmode
\textbf{loadstring} (\emph{string}) -- XML data to load Student info

\end{description}\end{quote}
\index{courses (ps.Student attribute)}

\begin{fulllineitems}
\phantomsection\label{index:ps.Student.courses}\pysigline{\bfcode{courses}\strong{ = None}}
(course{[}{]}) Course list

\end{fulllineitems}

\index{filter\_courses() (ps.Student method)}

\begin{fulllineitems}
\phantomsection\label{index:ps.Student.filter_courses}\pysiglinewithargsret{\bfcode{filter\_courses}}{\emph{cutoff\_date}}{}
Sets each course's \emph{in\_progress} attribute to \textbf{True} if it is
between \emph{cutoff\_date} and today.
\begin{quote}\begin{description}
\item[{Parameters}] \leavevmode
\textbf{cutoff\_date} (\emph{datetime.date}) -- Courses between this day (inclusive) will be included.

\end{description}\end{quote}

\end{fulllineitems}

\index{first\_name (ps.Student attribute)}

\begin{fulllineitems}
\phantomsection\label{index:ps.Student.first_name}\pysigline{\bfcode{first\_name}\strong{ = None}}
(string) First name

\end{fulllineitems}

\index{gender (ps.Student attribute)}

\begin{fulllineitems}
\phantomsection\label{index:ps.Student.gender}\pysigline{\bfcode{gender}\strong{ = None}}
(string) Gender

\end{fulllineitems}

\index{gpa (ps.Student attribute)}

\begin{fulllineitems}
\phantomsection\label{index:ps.Student.gpa}\pysigline{\bfcode{gpa}\strong{ = None}}
(string) Grade point average

\end{fulllineitems}

\index{last\_name (ps.Student attribute)}

\begin{fulllineitems}
\phantomsection\label{index:ps.Student.last_name}\pysigline{\bfcode{last\_name}\strong{ = None}}
(string) Last name

\end{fulllineitems}

\index{load() (ps.Student method)}

\begin{fulllineitems}
\phantomsection\label{index:ps.Student.load}\pysiglinewithargsret{\bfcode{load}}{\emph{loadstring}}{}
Loads a Student with XML data received from loadstring.
\begin{quote}\begin{description}
\item[{Parameters}] \leavevmode
\textbf{loadstring} (\emph{string}) -- XML data to load Student from

\end{description}\end{quote}

\end{fulllineitems}

\index{to\_excel() (ps.Student method)}

\begin{fulllineitems}
\phantomsection\label{index:ps.Student.to_excel}\pysiglinewithargsret{\bfcode{to\_excel}}{}{}~
\begin{notice}{note}{Note:}
Requires that the Student has been loaded.
\end{notice}
\begin{quote}\begin{description}
\item[{Returns}] \leavevmode
Creates a xlwt Workbook of the Student's grades.

\end{description}\end{quote}

\end{fulllineitems}

\index{to\_json() (ps.Student method)}

\begin{fulllineitems}
\phantomsection\label{index:ps.Student.to_json}\pysiglinewithargsret{\bfcode{to\_json}}{}{}~
\begin{notice}{note}{Note:}
Requires that the Student has been loaded.
\end{notice}
\begin{quote}\begin{description}
\item[{Returns}] \leavevmode
Student in JSON format.

\end{description}\end{quote}

\end{fulllineitems}

\index{z\_to\_json() (ps.Student method)}

\begin{fulllineitems}
\phantomsection\label{index:ps.Student.z_to_json}\pysiglinewithargsret{\bfcode{z\_to\_json}}{\emph{strength=9}}{}~
\begin{notice}{note}{Note:}
Requires that the Student has been loaded.
\end{notice}

Does the same thing as to\_json, but compresses the information with
maximum strength before returning it.

\end{fulllineitems}


\end{fulllineitems}



\renewcommand{\indexname}{Python Module Index}
\begin{theindex}
\def\bigletter#1{{\Large\sffamily#1}\nopagebreak\vspace{1mm}}
\bigletter{p}
\item {\texttt{ps}}, \pageref{index:module-ps}
\end{theindex}

\renewcommand{\indexname}{Index}
\printindex
\end{document}
